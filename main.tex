%%%%%%%%%%%%%%%%%
% This is an example CV created using altacv.cls (v1.1, 21 November 2016) written by
% LianTze Lim (liantze@gmail.com), based on the 
% Cv created by BusinessInsider at http://www.businessinsider.my/a-sample-resume-for-marissa-mayer-2016-7/?r=US&IR=T
% 
%% It may be distributed and/or modified under the
%% conditions of the LaTeX Project Public License, either version 1.3
%% of this license or (at your option) any later version.
%% The latest version of this license is in
%%    http://www.latex-project.org/lppl.txt
%% and version 1.3 or later is part of all distributions of LaTeX
%% version 2003/12/01 or later.
%%%%%%%%%%%%%%%%

%% If you want to use \orcid or the
%% academicons icons, add "academicons"
%% to the \documentclass options. 
%% Then compile with XeLaTeX or LuaLaTeX.
% \documentclass[10pt,a4paper,academicons]{altacv}
\documentclass[10pt,a4paper]{altacv}

%% AltaCV uses the fontawesome and academicon fonts
%% and packages. 
%% See texdoc.net/pkg/fontawecome and http://texdoc.net/pkg/academicons for full list of symbols.
%% When using the "academicons" option,
%% Compile with LuaLaTeX for best results. If you
%% want to use XeLaTeX, you may need to install
%% Academicons.ttf in your operating system's font %% folder.


% Change the page layout if you need to
\geometry{left=1cm,right=9cm,marginparwidth=7.2cm,marginparsep=1.2cm,top=1cm,bottom=1cm}

% Change the font if you want to.

% If using pdflatex:
\usepackage[utf8]{inputenc}
\usepackage[T1]{fontenc}
\usepackage[default]{lato}

% If using xelatex or lualatex:
% \setmainfont{Lato}

% Change the colours if you want to
\definecolor{VividPurple}{HTML}{274bc4}
\definecolor{SlateGrey}{HTML}{2E2E2E}
\definecolor{LightGrey}{HTML}{666666}
\colorlet{heading}{VividPurple}
\colorlet{accent}{VividPurple}
\colorlet{emphasis}{SlateGrey}
\colorlet{body}{LightGrey}

% Change the bullets for itemize and rating marker
% for \cvskill if you want to
\renewcommand{\itemmarker}{{\small\textbullet}}
\renewcommand{\ratingmarker}{\faCircle}

%% sample.bib contains your publications
%\addbibresource{sample.bib}

\begin{document}
\name{Nicholas Lewanowicz}
  \tagline{Software Engineer}
% Cropped to square from https://en.wikipedia.org/wiki/Marissa_Mayer#/media/File:Marissa_Mayer_May_2014_(cropped).jpg, CC-BY 2.0
%\photo{2.5cm}{me}
\personalinfo{%
  % Not all of these are required!
  % You can add your own with \printinfo{symbol}{detail}
  \email{nicklewanowicz@gmail.com}
%   \phone{000-00-0000}
  \location{Ottawa, ON, Canada}
  \homepage{nicklewanowicz.github.io/}
  \twitter{NickLewanowicz}
  \linkedin{linkedin.com/in/nicklewanowicz}
%   \github{} % I'm just making this up though.
%   \orcid{orcid.org/0000-0000-0000-0000} % Obviously making this up too. If you want to use this field (and also other academicons symbols), add "academicons" option to \documentclass{altacv}
}

%% Make the header extend all the way to the right, if you want. Extend the right margin by 8cm (=6.8cm marginparwidth + 1.2cm marginparsep)
\begin{adjustwidth}{}{-8cm}
\makecvheader
\end{adjustwidth}

%% Provide the file name containing the sidebar contents as an optional parameter to \cvsection.
%% You can always just use \marginpar{...} if you do
%% not need to align the top of the contents to any
%% \cvsection title in the "main" bar.
\cvsection[page1sidebar]{Experience}

\cvevent{Front-End Developer }{Shopify }{09/2018 - Present}{Ottawa, ON}
\begin{itemize}
\item Design and implement new user flows and features related to payments on the Shopify platform
\item Interact with a GraphQL API with a React Frontend using TypeScript
\item Responsible for creating robust and resilient code and accompanying test-suites ready for production
\end{itemize}

\divider

\cvevent{Full-Stack Developer }{Accedian }{01/2018 - 09/2018}{Ottawa, ON}
\begin{itemize}
\item Design and implement efficient Artificial Intelligent algorithm to provide meaningful data with Go in a docker environment
\item Implementation of intuitive UI solutions for navigation of complex data analytics and modern UI design
\item Create appealing intuitive visualizations for big data applications
\end{itemize}

\divider

\cvevent{Front-End Developer }{Ciena }{05/2017 - 01/2018}{Ottawa, ON}
\begin{itemize}
\item Used D3 to implement visualizations of complex kernel density estimates and random forest machine learning models
\item Training new interns with the development environment and ensuring effective task completion
\item Designed and documented JSON API compliant RESTful api's
\end{itemize}

\divider

\cvevent{Software Developer }{Versaterm }{05/2017 - 09/2017}{Ottawa, ON}
\begin{itemize}
\item Integrated bleeding edge web frameworks into existing C\# and .NET stack to elevate the user experience of the application
\item Integration of UI features into mature legacy codebases
\item Designed api layer between the UI and SQL backend using JSON API
\end{itemize}

\divider

\cvevent{Full-Stack Developer }{Ciena }{09/2016 - 05/2017}{Ottawa, ON}
\begin{itemize}
\item Developed a full stack internal application to manage use of internal dependencies company wide using NodeJS, EmberJS, and MongoDB
\item Aid in the upgrading and implementation of new UI components and features to leverage ES6 improvements and features
\item Integrate new component features into existing company products
\end{itemize}


% \divider

% \cvevent{Product Engineer}{Google}{23 June 1999 -- 2001}{Palo Alto, CA}

% \begin{itemize}
% \item Joined the company as employe \#20 and female employee \#1
% \item Developed targeted advertisement in order to use user's search queries and show them related ads
% \end{itemize}

% ##Commented Day of My Lify
% \cvsection{A Day of My Life}

% % Adapted from @Jake's answer from http://tex.stackexchange.com/a/82729/226
% % \wheelchart{outer radius}{inner radius}{
% % comma-separated list of value/text width/color/detail}
% \wheelchart{1.5cm}{0.5cm}{%
%   10/13em/accent!30/Sleeping \& dreaming about research, 
%   25/9em/accent!60/Attending Lectures \& Taking Notes,
%   5/12em/accent!10/Programming \& Hacking, 
%   20/12em/accent!40/Social Media,
%   5/8em/accent!20/Group Study Sessions (Never too much studying!),
%   30/9em/accent/Presenting Research or working on projects,
%   5/8em/accent!20/Playing with my dog Max
% }

% \clearpage
%## Commented Undergrad research section
% \cvsection[page2sidebar]{Undergraduate Research}

% \nocite{*}

% \printbibliography[heading=pubtype,title={\printinfo{\faBook}{Review Articles}},type=book]

% \divider

% \printbibliography[heading=pubtype,title={\printinfo{\faFileTextO}{Course Articles}}, type=article]

% \divider

% \printbibliography[heading=pubtype,title={\printinfo{\faGroup}{Poster Presentations}},type=article]

\end{document}
